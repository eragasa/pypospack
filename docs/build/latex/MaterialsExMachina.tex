%% Generated by Sphinx.
\def\sphinxdocclass{report}
\documentclass[letterpaper,10pt,english]{sphinxmanual}
\ifdefined\pdfpxdimen
   \let\sphinxpxdimen\pdfpxdimen\else\newdimen\sphinxpxdimen
\fi \sphinxpxdimen=.75bp\relax

\PassOptionsToPackage{warn}{textcomp}
\usepackage[utf8]{inputenc}
\ifdefined\DeclareUnicodeCharacter
 \ifdefined\DeclareUnicodeCharacterAsOptional
  \DeclareUnicodeCharacter{"00A0}{\nobreakspace}
  \DeclareUnicodeCharacter{"2500}{\sphinxunichar{2500}}
  \DeclareUnicodeCharacter{"2502}{\sphinxunichar{2502}}
  \DeclareUnicodeCharacter{"2514}{\sphinxunichar{2514}}
  \DeclareUnicodeCharacter{"251C}{\sphinxunichar{251C}}
  \DeclareUnicodeCharacter{"2572}{\textbackslash}
 \else
  \DeclareUnicodeCharacter{00A0}{\nobreakspace}
  \DeclareUnicodeCharacter{2500}{\sphinxunichar{2500}}
  \DeclareUnicodeCharacter{2502}{\sphinxunichar{2502}}
  \DeclareUnicodeCharacter{2514}{\sphinxunichar{2514}}
  \DeclareUnicodeCharacter{251C}{\sphinxunichar{251C}}
  \DeclareUnicodeCharacter{2572}{\textbackslash}
 \fi
\fi
\usepackage{cmap}
\usepackage[T1]{fontenc}
\usepackage{amsmath,amssymb,amstext}
\usepackage{babel}
\usepackage{times}
\usepackage[Bjarne]{fncychap}
\usepackage{sphinx}

\usepackage{geometry}

% Include hyperref last.
\usepackage{hyperref}
% Fix anchor placement for figures with captions.
\usepackage{hypcap}% it must be loaded after hyperref.
% Set up styles of URL: it should be placed after hyperref.
\urlstyle{same}

\addto\captionsenglish{\renewcommand{\figurename}{Fig.}}
\addto\captionsenglish{\renewcommand{\tablename}{Table}}
\addto\captionsenglish{\renewcommand{\literalblockname}{Listing}}

\addto\captionsenglish{\renewcommand{\literalblockcontinuedname}{continued from previous page}}
\addto\captionsenglish{\renewcommand{\literalblockcontinuesname}{continues on next page}}

\addto\extrasenglish{\def\pageautorefname{page}}

\setcounter{tocdepth}{0}



\title{Materials Ex Machina Documentation}
\date{May 25, 2019}
\release{2017.7.22}
\author{Eugene J. Ragasa}
\newcommand{\sphinxlogo}{\vbox{}}
\renewcommand{\releasename}{Release}
\makeindex

\begin{document}

\maketitle
\sphinxtableofcontents
\phantomsection\label{\detokenize{index::doc}}


Deus Ex Machina. (1) a Latin phrase originally described an ancient plot device used in Greek and Roman theatre to resolve complicated or even hopeless situations in the plots of their plays. Literally means “god from the machine”, (2) a person or thing that appears or is introduced suddenly and unexpectedly and provides a contrived solution to an insoluble difficulty.

Materials Ex Machina represents the current state of computational materials science in a variety of different ways.  Certain tools such as Density Functional Theory are being used as exhaustive search tools for finding new materials from techniques generally referred to as in silico high-throughput computing.  Computational searches for optimal alloys using tools such as AT-AT provide impetus for a computational driven approach to new materials design.

Computational material science is also a difficult field because much of the information to do high quality simulations simply does not exist, and approaches to creating high quality simulations is usually not documented in peer-reviewed journal articles, tutorials usually only cover trivial material problems, and reference textbooks concentrate on theory rather the ability to do useful simulations.

Within this corner of the Internet, I’m posting a variety of different notes and manuscripts distilled from a variety of notes from the University of Florida.  These notes are combined from classes, research, personal projects, and personal interests.  My research is currently in computational materials science, and these online notes began when I started my doctoral program, but I’m consolidating the notes I’m using for my dissertation and code documentation here.

My research is focused on the use of the following tools:
\begin{itemize}
\item {} 
{\hyperref[\detokenize{dft/index:dft}]{\sphinxcrossref{\DUrole{std,std-ref}{Density Functional Theory.}}}}

\item {} 
{\hyperref[\detokenize{md/index:md}]{\sphinxcrossref{\DUrole{std,std-ref}{Molecular Dynamics.}}}}

\item {} 
{\hyperref[\detokenize{ld/index:ld}]{\sphinxcrossref{\DUrole{std,std-ref}{Lattice Dynamics.}}}}

\end{itemize}


\chapter{Computational Materials Science Bootcamp}
\label{\detokenize{index:computational-materials-science-bootcamp}}
Contents:


\section{About}
\label{\detokenize{about:id1}}\label{\detokenize{about::doc}}

\subsection{Eugene J. Ragasa}
\label{\detokenize{about:eugene-j-ragasa}}

\subsubsection{Education}
\label{\detokenize{about:education}}
PhD, Materials Science and Engineering, University of Florida

MSES, Mechanical Engineering, University of the Pacific

MA, Mathematical Finance, Columbia University

BS, Mathematics, United States Military Academy


\subsubsection{Publications}
\label{\detokenize{about:publications}}
Has not been done yet.


\subsubsection{Presentations}
\label{\detokenize{about:presentations}}

\subsubsection{Raman Spectra Simulation (Unpublished work)}
\label{\detokenize{about:raman-spectra-simulation-unpublished-work}}
The effect of anion intrinsic defects of Raman spectra on fluorite-structure materials with a particular focus on CaF2, CeO2, and UO2.  Since CeO2 and UO2 require DFT+U corrections perturbational methods aren’t available to calculate Raman signatures.  Instead the derivatives of the polarization tensor is calculated through the application of finite electric fields.  DFT+U has the problem of creating metastable states which need to be resolved to find the ground state energy requiring the use of techniques such as occupation matrix control, +U ramping, and annealing.  The value of this research for experimental researchers to resolve changes in Raman signatures to particular point defects.  For computational scientists, this research will also explore techniques for resolving metastable states to the ground state, particuarly with the application of an electric field to the calculations.


\section{Computational Materials Bootcamp}
\label{\detokenize{bootcamp/index:computational-materials-bootcamp}}\label{\detokenize{bootcamp/index::doc}}
This bootcamp was developed over a period of years to provide a baseline understanding of computer skills required for people new to the field of computational materials science.  This bootcamp assumes that you have access to a UNIX based computational cluster.


\section{Computational Simulation Tools}
\label{\detokenize{computational_simulation_tools:computational-simulation-tools}}\label{\detokenize{computational_simulation_tools::doc}}

\subsection{Contents}
\label{\detokenize{computational_simulation_tools:contents}}

\subsubsection{Crystallography}
\label{\detokenize{crystallography/index:crystallography}}\label{\detokenize{crystallography/index:crystal-lattice}}\label{\detokenize{crystallography/index::doc}}
Crystallography deals with the arrangement of atoms in a crystaline solid.
The approach to crystallography provided aimed toward computational materials.
Since computational materials often require the development of computer codes, the approach provided includes what I feel is an appropriate mathematical approach.


\paragraph{Contents}
\label{\detokenize{crystallography/index:contents}}

\subparagraph{The Perfect Crystal and the Simulation Cell}
\label{\detokenize{crystallography/perfect_crystal:the-perfect-crystal-and-the-simulation-cell}}\label{\detokenize{crystallography/perfect_crystal::doc}}

\subparagraph{The Perfect Crystal}
\label{\detokenize{crystallography/perfect_crystal:the-perfect-crystal}}
Inorganic materials are typically crystalline, meaning that they are periodic at the atomic scale.
A crystal consists of atoms arranged in a pattern that repeats periodically in three dimensions.  In this defiinition, the pattern can consist of a single atom, a group of atoms, a molecule or group of molecules.  As a solid forms, atoms and/or molecules asume fixed orientations and positions with respect to each other.  As a necessary consequence of particle growth, atoms and molecules position themselves as to minimize the forces acting upon it.  Each molecule entering the solid phase is influenced in almost exactly the same way as the proceeding molecule, and the solid particle consists of three-dimensional ordered arrray of molecules; that is a crystal.

A crystal is a periodic array of atoms and consists of a set of lattice vectors which all of Euclidean space and an atomic basis, which when combined form a unit cell.  For computational materials, it is convenience to represent crystals as a material with infinite extent, but with a finite representation for computational tractibility.  The coordinate system in which to represent an infinite solid.

See books by

\phantomsection\label{\detokenize{crystallography/perfect_crystal:id1}}{\hyperref[\detokenize{crystallography/perfect_crystal:sands1969-book-crystals}]{\sphinxcrossref{{[}San69{]}}}} Donald Sands. \sphinxhref{https://www.amazon.com/Introduction-Crystallography-Dover-Books-Chemistry-ebook/dp/B008TVLYUC/ref=sr\_1\_2?ie=UTF8\&qid=1502437192\&sr=8-2\&keywords=crystallography}{Introduction to Crystallography}.  This book is currently published by Dover publications and is a rather cheap book on the subject.

\phantomsection\label{\detokenize{crystallography/perfect_crystal:id2}}{\hyperref[\detokenize{crystallography/perfect_crystal:sands2002-book-crystals-2}]{\sphinxcrossref{{[}San02{]}}}}. Donald Sands. \sphinxhref{https://www.amazon.com/Vectors-Tensors-Crystallography-Donald-Sands/dp/0201071479/ref=sr\_1\_1?ie=UTF8\&qid=1502437679\&sr=8-1\&keywords=vectors+and+tensors+in+crystallography}{Vectors and Tensors in Crystallography}.  This is a much more mathematical treatment of crystallography.

\phantomsection\label{\detokenize{crystallography/perfect_crystal:id3}}{\hyperref[\detokenize{crystallography/perfect_crystal:hammond2001-book-crystal}]{\sphinxcrossref{{[}HH01{]}}}} Christopher Hammond. \sphinxhref{https://www.amazon.com/Basics-Crystallography-Diffraction-Fourth-International/dp/0198738684/ref=sr\_1\_5?ie=UTF8\&qid=1502436584\&sr=8-5\&keywords=crystallography}{Basics of Crystallography}

\phantomsection\label{\detokenize{crystallography/perfect_crystal:id4}}{\hyperref[\detokenize{crystallography/perfect_crystal:giacovazzo2002-book-crystal}]{\sphinxcrossref{{[}Gia02{]}}}} Giacovazzo and Monaco. \sphinxhref{https://www.amazon.com/Fundamentals-Crystallography-C-Giacovazzo/dp/0198509588/ref=sr\_1\_3?s=books\&ie=UTF8\&qid=1502436703\&sr=1-3\&keywords=giacovazzo}{Fundamentals of Crystallography}


\subparagraph{A Graphical Explanation of a Lattice}
\label{\detokenize{crystallography/perfect_crystal:a-graphical-explanation-of-a-lattice}}
We will replace the traditional terminology of the unit cell with a mathematical description which is more suitable for computational materials.  The notation for computational materials isn’t standardized, but the terminology selected here is consistent with computational materials research, and the exposition here is one for practical notation and amenable for the application of applied mathematics and understanding computational underpinings.

In conventional notation, the lattice vectors are often referred to as \(\mathbf{a}\), \(\mathbf{b}\), and \(\mathbf{c}\), and are often conveniently described with the length of the lattice vectors, \(a\), \(b\), and \(c\), to represent the length of the vectors \(\mathbf{a}\), \(\mathbf{b}\), and \(\mathbf{c}\), respectively.  The angles \(\alpha\), \(\beta\), and \(\gamma\) are used to describe the angles between the lattice vectors.


\subparagraph{A Mathematical Explanation of a Lattice}
\label{\detokenize{crystallography/perfect_crystal:a-mathematical-explanation-of-a-lattice}}
Computationally, these are an inelegant representation of a crystal lattice and use the following notation instead.  These lattice vectors are denoted \(\mathbf{a}_1\), \(\mathbf{a}_2\), and \(\mathbf{a}_3\).  These vectors are often denoted as the row vectors \(\mathbf{a}_{i} = [a_{i1}, a_{i2}, a_{i3}]\), but are often more useful in matrix operations as column vectors.  Whether or not the lattice vectors should be used as a column or row vector, should be obvious by the application.

In computational materials, these lattice vectors are often represented as the matrix \(\mathbf{H}\), where
\begin{equation*}
\begin{split}\mathbf{H}
=
\begin{bmatrix}
    a_{11} & a_{12} & a_{13} \\
    a_{21} & a_{22} & a_{23} \\
    a_{31} & a_{32} & a_{33}
\end{bmatrix}
=
\begin{bmatrix}
    \mathbf{a}_1 \\
    \mathbf{a}_2 \\
    \mathbf{a}_3
\end{bmatrix}\end{split}
\end{equation*}

\subparagraph{Cartesian Coordinate System and Direct Lattice Coordinates}
\label{\detokenize{crystallography/perfect_crystal:cartesian-coordinate-system-and-direct-lattice-coordinates}}
There are two types of representation of the basis of atoms in computational material system.  One represents the basis of atoms in the unit cell using the standard unit vectors in Eucliean space, which is commonly known as the Cartesian coordinate sytem.  The second in the representation of the lattice vectors of the atom.

Given the representation of \(\mathbf{H}\) as a matrix containing the lattice vectors, then the transpose of that matrix is \(\mathbf{H}^T\).  The transformation of the direct coordinates to cartesian coordinates is
\begin{equation*}
\begin{split}\begin{bmatrix} x \\ y \\ z \end{bmatrix} = \mathbf{H}^T \begin{bmatrix} u \\ v \\ w \end{bmatrix}\end{split}
\end{equation*}
where \(x\), \(y\), and \(z\) are elements of a point in a Cartesian coordinate system and \(u\), \(v\), and \(w\) are elements of the same points defined by the edge vectors of the lattice.
\begin{equation*}
\begin{split}\begin{bmatrix}
     x\\ y \\ z
\end{bmatrix}
=
\begin{bmatrix}
    a_{11} & a_{21} & a_{31} \\
    a_{12} & a_{22} & a_{32} \\
    a_{13} & a_{23} & a_{33}
\end{bmatrix}
\begin{bmatrix}
    u \\ v \\ w
\end{bmatrix}\end{split}
\end{equation*}
The transformation from the Cartesian coordinate system to the direct lattice is
\begin{align*}\!\begin{aligned}
\begin{bmatrix} x \\ y \\ z \end{bmatrix} = \mathbf{H}^T \begin{bmatrix} u \\ v \\ w \end{bmatrix}\\
\mathbf{H}^T)^{-1} \begin{bmatrix} x \\ y \\ z \end{bmatrix} = \begin{matrix} u \\ v \\ w \end{matrix}\\
\end{aligned}\end{align*}

\subparagraph{The Simulation Cell}
\label{\detokenize{crystallography/perfect_crystal:the-simulation-cell}}
The simulation cell consists of the lattice vectors and a basis of atoms in that lattice vector.  Atoms of different species must be identified.  From a notational perspective, differentiation of the chemical species uses small Greek subscripts to identify different species as a general terminology (e.g. \(\mathbf{r}_{\alpha}\) or the chemical symbols for specificity (e.g. \(\mathbf{r}_{Cu}\)).

The atomic basis consists of the atoms which are contained within the volume bounded by the lattice vectors.  From a programatic implementation, these basis atoms are implemented as an Atom class which as the properties: symbol, position, index, and variety of specific properties (such as magnetic moment).


\subparagraph{Lattice Vectors and Periodic Boundary Conditions}
\label{\detokenize{crystallography/perfect_crystal:lattice-vectors-and-periodic-boundary-conditions}}
In order to motivate the idea of an infinite solid mathematically, let us start with a function which is periodic in one dimension.  A function \(f\) is periodic with period \(P\) if
\begin{equation*}
\begin{split}f(x+P) = f(x)\end{split}
\end{equation*}
where the periodic function can be defined as a function whose graph exhibits translational symmetry.  The most ubiqitious of these function are the trigonmetric series which are integral to calculating and modelling solid matter.  The sine function is periodic
\begin{equation*}
\begin{split}\sin(x+2\pi) = \sin(x)\end{split}
\end{equation*}
This concept of periodicity is also referred to as translational invariance.  The set of points due to translational symmetry from the infinite discrete set
\begin{equation*}
\begin{split}\{ x + n P \}\end{split}
\end{equation*}
In three-dimensions, translational symmetry are in the direction of lattice vectors.  If periodic boundary conditions are applied in three directions, the three lattice vectors must span all of Euclidean 3D space.

We then motivate the discussion through molecular dynamics and the strength in molecular dynamics in solving dynamical systems by sampling of the Boltzmann distribution through a variety of thermodynamic ensembles by looking at the Carr-Parrinello method.  Due to the computational cost of a DFT calculation between the updates in the Carr-Parrinello method, the need for empirical interatomic potentials becomes motivated.  We discuss different classes of interatomic potentials.


\subparagraph{Additional References}
\label{\detokenize{crystallography/perfect_crystal:additional-references}}

\subparagraph{References}
\label{\detokenize{crystallography/perfect_crystal:references}}



\subparagraph{Reciprocal Lattice}
\label{\detokenize{crystallography/reciprocal_lattice:reciprocal-lattice}}\label{\detokenize{crystallography/reciprocal_lattice:id1}}\label{\detokenize{crystallography/reciprocal_lattice::doc}}
The reciprocal lattice represents the Fourier transform of another lattice from real space into the momentum.


\subparagraph{Pontryagin Duality}
\label{\detokenize{crystallography/reciprocal_lattice:pontryagin-duality}}
In harmonic analysis and theory of topological groups,  Pontrygin duality explains the general properties of
the Fourier transform on locally compact grousp.


\subparagraph{Defect Energy Calculations}
\label{\detokenize{crystallography/defect_energy:defect-energy-calculations}}\label{\detokenize{crystallography/defect_energy::doc}}

\subparagraph{Surfaces}
\label{\detokenize{crystallography/surface_energy:surfaces}}\label{\detokenize{crystallography/surface_energy::doc}}
Knowledge of the geometrical arrangement of the atoms near the surface is a basic ingredient for the study of structural and dynamic properties of a metal surface.  In the surface geometry, surface tension and stress play important roles in determining whether and how surfaces reconstruct and relax.

Cosider the atoms in the bulk and surface regions of a crytal.  In the bulk, atoms possess lower energy because they are tightly bound.  At the surface, the atoms posess higher energy because they are less tightly bound.  The sum of all the excess energies of the surface atoms is the surface energy.


\subparagraph{Stacking faults in fcc metals}
\label{\detokenize{crystallography/fcc_sf:stacking-faults-in-fcc-metals}}\label{\detokenize{crystallography/fcc_sf:fcc-stacking-faults}}\label{\detokenize{crystallography/fcc_sf::doc}}

\subparagraph{Calculation of intrinsic stacking fault energies}
\label{\detokenize{crystallography/fcc_sf:calculation-of-intrinsic-stacking-fault-energies}}
Notes from this section are fro Chandran and Sondhi {[}chandran2001\_Ni\_sf{]}.

The {[}111{]} intrinsic stacking fault (ISF) in the fcc structure is created by removing one layer, which changes the sequence from ABCA to ABCB.
The \(\gamma_{ISF}\) is formally defined as
\begin{equation*}
\begin{split}\gamma_{ISF} = \frac{1}{A}(F_{ISF} - F)\end{split}
\end{equation*}
where \(F_{ISF}\) and \(F\) are the free-energies of the structure with and without a stacking fault, respectively, and \(A\) is the area of the faulted region.  When these calculations are done at \(T=0\), the free energies and be replaces with \(E_{ISF}\) and \(E\).

The energies \(E_{ISF}\) and \(E\) can be calculated using first-principle methods using the supercell method.

ABCABCABC

ABC\textbar{}BCA\textbar{}CAB\textbar{}

ABCABCA\textbar{}CABCABC


\subparagraph{Extrinsic Stacking Fault}
\label{\detokenize{crystallography/fcc_sf:extrinsic-stacking-fault}}



\paragraph{Ideal Crystal}
\label{\detokenize{crystallography/index:ideal-crystal}}
{\hyperref[\detokenize{crystallography/perfect_crystal:perfect-crystal}]{\sphinxcrossref{\DUrole{std,std-ref}{The Perfect Crystal}}}}


\subparagraph{Stacking Faults}
\label{\detokenize{crystallography/index:stacking-faults}}
A stacking fault is a type of defect which characterizes the disordering of crystallographic planes.

The stacking fault energy (SFE) is a fundamental parameter in the design of alloys.
It has a large impact on the creep properties of materials in temperature and stress ranges where dominant deformation is by dislocation glide.  Low SFE leads to a large separation between Shockley dislocation which hinders cross-slip and thereby effectively reduces the creep rate.

\sphinxtitleref{Stacking faults in fcc materials \textless{}fcc\_stacking\_faults\textgreater{}}


\paragraph{Further Reading}
\label{\detokenize{crystallography/index:further-reading}}
These notes have been developed from such a large number of sources which provides a different exposition of the same topic.
\begin{itemize}
\item {} 
\phantomsection\label{\detokenize{crystallography/index:id1}}{\hyperref[\detokenize{crystallography/index:id3}]{\sphinxcrossref{{[}San69{]}}}}.  Sands, D.E., \sphinxhref{https://www.amazon.com/Introduction-Crystallography-Dover-Books-Chemistry/dp/0486678393}{Introduction to Crystallography} by Donald E. Sands. \phantomsection\label{\detokenize{crystallography/index:id2}}{\hyperref[\detokenize{crystallography/index:id3}]{\sphinxcrossref{{[}San69{]}}}}

\end{itemize}


\paragraph{Bibliography}
\label{\detokenize{crystallography/index:bibliography}}





\subsubsection{Density Functional Theory}
\label{\detokenize{dft/index:density-functional-theory}}\label{\detokenize{dft/index::doc}}
The purpose of this page is to provide some introduction and a place to get more information.  As I write my dissertation, I will begin to slowly increase the amount of information provided, include both references, manuscripts, and some tutorials.  I have found that the introduction to DFT has been somewhat frustrating when I started by graduate career because much of the theory is confusing, which might be surprising to some because I have a background in mathematics before getting into computational materials science.


\paragraph{Additional Resources}
\label{\detokenize{dft/index:additional-resources}}\begin{itemize}
\item {} 
\sphinxhref{http://vergil.chemistry.gatech.edu/notes/}{David Sherrill’s Notes on Computational Chemistry}

\item {} 
Klaus Capelle. \sphinxhref{http://simons.hec.utah.edu/TSTCsite/SimonsProblems/Density-Functional\%20Theory.pdf}{A Bird’s-Eye View of Density Functional Theory}

\end{itemize}


\paragraph{Contents}
\label{\detokenize{dft/index:contents}}

\subparagraph{Introduction to Density Functional Theory}
\label{\detokenize{dft/dft_intro:introduction-to-density-functional-theory}}\label{\detokenize{dft/dft_intro::doc}}
Density Functional Theory (DFT) is a quantum mechanical technique which solves the ground-state wavefunction of a many electron problem by solving electron density of the system and relying on the one-to-one correspondence between the electron density and the wavefunction.

The discussion of the subject here is not a rigorous explanation, but references to a more rigorous explanation for those interested in the subject is provided, including both original citations as well more modern references.  Instead, an exposition which highlights fundamental assumptions and mathematical concepts are discussed to give an understanding of the sources of error due to convergence issues and fundamental sources of errors due to inadequate modelling assumptions.

To motivate the discussion of Density Functional Theory (DFT), let us first start with a discussion of quantum mechanics and solutions of Schroedingers equation to introduce the notation familiar with quantum mechanics.  We explain the components of the Schroedinger’s equation as applied to to nuclear-electronic solutions pertinent to materials science.

The adibiatic approximation of the nuclear-electron problem allows the separation of the nuclear wavefunction from the electronic wavefunctions.  The application of the Born-Oppenheimer approximation allows to solve the electronic wavefunction separately by treating the electrons where clamped nuclear positions produces potential which interacts with the electrons.

Next successive solutions of the these solutions are discussed with particular discussion to solutions of the Hartree-Roothan equations and Slater polynomials.

We then discuss the foundational Hohenberg-Kohn theorems and the Kohn-Sham equations which are the foundation of DFT.  A discussion of solutions of the Kohn-Sham using orbital basis sets (more common for modelling molecules) and the plane wave basis sets (more common in modelling solids) and the ramification for computation.

Finally a discussion on the psuedopotentials, GGA and PBE.


\subparagraph{Schroedinger equation}
\label{\detokenize{dft/dft_intro:schroedinger-equation}}

\subparagraph{Single-particle}
\label{\detokenize{dft/dft_intro:single-particle}}
It is useful to step back and recall elementary quantum mechanics to both review concept and to introduce terminology required to understand Density Functional Theory.  Let us consider the Schroedinger’s equation for a single-particle.
\begin{equation*}
\begin{split}\hat{H}\Psi(\bm{r}),t) = i \hbar \frac{\partial}{\partial t} \Psi(\mathbf{r},t)\end{split}
\end{equation*}
The Hamiltonian.

The wavefunction.


\subparagraph{Multi-particle}
\label{\detokenize{dft/dft_intro:multi-particle}}
\sphinxurl{http://www.reed.edu/physics/courses/P342.S10/Physics342/page1/files/Lecture.31.pdf}
In quantum mechanics, the information about a system is contained in a system’s wavefunction \(\Psi\).  Here we are concerned with the electronic structure of atoms, molecules, and solids.  The degrees of freedom of the nuclei appear only in the form of a potential \(\nu(\mathbf{r})\), so that the function depends only on the electronic coordinates, from the Born-Oppenheimer approximation which allows to use the time-independent version of the Schroedinger’s equation.

Zwiebach’s notes on
\sphinxhref{https://ocw.mit.edu/courses/physics/8-05-quantum-physics-ii-fall-2013/lecture-notes/MIT8\_05F13\_Chap\_01.pdf}{wave mechanics},
\sphinxhref{https://ocw.mit.edu/courses/physics/8-05-quantum-physics-ii-fall-2013/lecture-notes/MIT8\_05F13\_Chap\_02.pdf}{bras and kets},
\sphinxhref{https://ocw.mit.edu/courses/physics/8-05-quantum-physics-ii-fall-2013/lecture-notes/MIT8\_05F13\_Chap\_03.pdf}{linear algebra},
\sphinxhref{https://ocw.mit.edu/courses/physics/8-05-quantum-physics-ii-fall-2013/lecture-notes/MIT8\_05F13\_Chap\_02.pdf}{Dirac noation},

Consider a system with Hamiltonian \(\hat{H}\) and the time independent Schrodinger equation
\begin{equation*}
\begin{split}\hat{H}\Psi = E\Psi\end{split}
\end{equation*}
by using the variational principle
\begin{equation*}
\begin{split}E_{gs} = \frac{\bra{\Psi}\hat{H}\ket{Psi}}
              {\bra{\Psi}\ket{\Psi}}\end{split}
\end{equation*}
Is an eigenvalue problem and be expressed through an eigenvalue eigenvector expansion

Is an eigenvalue eigenvalue expansion.
\begin{equation*}
\begin{split}H\ket{\psi_n}=\epsilon\ket{\psi_n}\end{split}
\end{equation*}

\subparagraph{Examples}
\label{\detokenize{dft/dft_intro:examples}}
\DUrole{xref,std,std-ref}{Numerical solution of Wood-Saxon potential}


\subparagraph{Born-Oppenheimer Approximation}
\label{\detokenize{dft/dft_intro:born-oppenheimer-approximation}}
Consider a system consisting of atoms.  Let us denote the collection of the position of atoms by the set \(\left\{ \mathbf{R}_n \right\}\), and the collection of position of electrons by the set \(\left\{ \mathbf{r}_i \right\}\).

The Hamiltonian for this system is
\begin{equation*}
\begin{split}\hat{H} = \hat{T}_e + \hat{T}_n + V\left( \left\{ \mathbf{r}_i \right\},
                                          \left\{ \mathbf{R}_i \right\} \right)\end{split}
\end{equation*}
Then ground state wavefunction, \sphinxtitleref{:math:}Phi = Phileft(left\{mathbf\{r\}\_iright\},left\{mathbf\{R\}\_nright\}right){}`, which solves the Schrodinger equation, \(\hat{H}\Phi=E\Phi\).


\subparagraph{Additional Resources}
\label{\detokenize{dft/dft_intro:additional-resources}}
For standard references on quantum mechanics, the undergraduate textbook by Griffiths \phantomsection\label{\detokenize{dft/dft_intro:id1}}{\hyperref[\detokenize{dft/dft_intro:griffiths1995-book-qm}]{\sphinxcrossref{{[}Gri16{]}}}}, and the graduate textbooks by Shankar \phantomsection\label{\detokenize{dft/dft_intro:id2}}{\hyperref[\detokenize{dft/dft_intro:shankar2012-book-qm}]{\sphinxcrossref{{[}Sha12{]}}}}, and Sakurai \phantomsection\label{\detokenize{dft/dft_intro:id3}}{\hyperref[\detokenize{dft/dft_intro:sakurai2014-book-qm}]{\sphinxcrossref{{[}SN14{]}}}}.

For an introduction to DFT, a gentle introduction to the subject which only requires a cursory understanding of quantum mechanics can be found in \sphinxhref{http://simons.hec.utah.edu/TSTCsite/SimonsProblems/Density-Functional\%20Theory.pdf}{Capelle} \phantomsection\label{\detokenize{dft/dft_intro:id4}}{\hyperref[\detokenize{dft/dft_intro:capelle2006-dft}]{\sphinxcrossref{{[}Cap06{]}}}}

Since the original paper on the Born-Oppenheimer approximation is in German, it is necessary to have an English language resource.


\subparagraph{Bibliography}
\label{\detokenize{dft/dft_intro:bibliography}}



\subparagraph{Orbital Density Functional Theory}
\label{\detokenize{dft/dft_orbital:orbital-density-functional-theory}}\label{\detokenize{dft/dft_orbital:dft-orbital}}\label{\detokenize{dft/dft_orbital::doc}}
Some information on orbital Density Functional theory like Gaussian.


\subparagraph{Projector Augmented Wave Density Functional Theory}
\label{\detokenize{dft/dft_paw:projector-augmented-wave-density-functional-theory}}\label{\detokenize{dft/dft_paw:dft-paw}}\label{\detokenize{dft/dft_paw::doc}}
Let us motivate our discussion of the specifics of projector augmented wave density functional theory (PAW-DFT) {[}{]}{[} by revisiting the implication of crystalline arrangements of atoms on the electron density. Suppose we have the real space lattice vectors: :math:{]}mathbf\{a\}\_1{}`, \(\mathbf{a}_2\), and \(\mathbf{a}_3\), which produces the Bravais lattice \(\mathbf{R}_{\mathbf{n}}\), where \(\mathbf{n}=\left[n_1,n_2,n_3\right]\).
\begin{equation*}
\begin{split}\mathbf{R}_{\mathbf{n}} = n_1 \mathbf{a}_1 + n_2 \mathbf{a}_2 + n_3 \mathbf{a}_3\end{split}
\end{equation*}
Due to periodicity, all functions dependent upon the Bravais lattice must have a periodic representation.  This includes the electronic density in an atomic crystal \(\rho(\mathbf{r})\) which can be written as a periodic function
\begin{equation*}
\begin{split}\rho(\mathbf{r}) = \rho(\mathbf{R}_{\mathbf{n}} + \mathbf{r})\end{split}
\end{equation*}
And it is useful to represent it as a Fourier series expansion, an expansion in sines and cosines.
\begin{equation*}
\begin{split}\rho(\mathbf{r}) = \sum_m \rho_m e^{i \mathbf{G}_m \cdot \mathbf{r}} + e^{i \mathbf{G}_m \cdot \mathbf{R}_n}\end{split}
\end{equation*}
Since the function \(\rho(\mathbf{r})\) is periodic, then for any choice of \(\mathbf{n},\mathbf{k} \in \mathbb{R}\)

Then the reciprocal lattice vectors are defined


\subparagraph{Bloch’s Theorem}
\label{\detokenize{dft/dft_paw:bloch-s-theorem}}
The electron wavefunctions of a crystal have a basis consisting entirely of Bloch wave energy eigenstates.

The energy eigenstates for an electron in a crystal can be written as Bloch waves.

A wavefunction \(\Psi\) is a Bloch wave if it has the form:
\begin{equation*}
\begin{split}\Psi(\mathbf{r}) = e^{ik \cdot r} u(\mathbf{r})\end{split}
\end{equation*}

\subparagraph{Energy Cutoff}
\label{\detokenize{dft/dft_paw:energy-cutoff}}
For a DFT calculation with plane waves, the electronic wavefunction is represented as the infinite summation of plane waves, which mus be truncated to a finite series.

With more plane waves, there is more accuracy, but also at higher computational cost.

Plane waves with less kinetic energy
\begin{equation*}
\begin{split}\frac{\hbar^2}{2m}\lvert \bm{k} + \bm{G} \rvert^2\end{split}
\end{equation*}
have a higher contribution to the sum, so the plane waves with lower energy have the highest contribution.

the deermination of the energy cutoff.
Detemining of the energy cutoff for the plane wave basis set expansion has a large effect on the cost of calculation as well as the accuracy of calculation.  If \(E_{cut}\) is the energy cutoff, then plane waves with kinetic energy less than \(E_{cut}\) are excluded from the basis set.
\begin{equation*}
\begin{split}\lvert \mathbf{G}+\mathbf{k} \rvert < G_{cut}\end{split}
\end{equation*}
where
\begin{equation*}
\begin{split}E_{cut}=\frac{\hbar}{2m}G^2_{cut}\end{split}
\end{equation*}

\subparagraph{Additional Reading}
\label{\detokenize{dft/dft_paw:additional-reading}}\begin{itemize}
\item {} 
\sphinxhref{https://github.com/certik/sphinx-jax/blob/master/src/quantum/paw.rst}{Projector Augmented-Wave Method (PAW)}

\item {} 
{\color{red}\bfseries{}{}`}The Projector Augmented-wave method \textless{}\sphinxurl{https://arxiv.org/pdf/0910.1921v2.pdf}\textgreater{}\_ by Carsten Rostgaard

\end{itemize}


\subparagraph{References}
\label{\detokenize{dft/dft_paw:references}}

\paragraph{Bibliography}
\label{\detokenize{dft/index:bibliography}}

\subsubsection{Empirical Interatomic Potentials}
\label{\detokenize{eip/index:empirical-interatomic-potentials}}\label{\detokenize{eip/index::doc}}
These are some notes on empirical interatomic potentials.


\paragraph{Potentials on covalent solids}
\label{\detokenize{eip/index:potentials-on-covalent-solids}}
For semiconductors and other covalent solids modelling is a more formidible problem.

Tersoff {[}tersoff1986\_si\_pot{]}


\paragraph{Contents}
\label{\detokenize{eip/index:contents}}

\subparagraph{Fitting Empirical Interatomic Potentials}
\label{\detokenize{eip/fitting:fitting-empirical-interatomic-potentials}}\label{\detokenize{eip/fitting::doc}}
This is going to be some information on fitting interatomic potentials.  Let us first start with the purpose of fitting a potential.  A empirical interatomic potential is a mathematical model which maps an atomic structure to an energy.


\subparagraph{Buckingham Potentials}
\label{\detokenize{eip/buckingham:buckingham-potentials}}\label{\detokenize{eip/buckingham::doc}}
Some notes on buckingham potentials


\subparagraph{Embedded Atom Model}
\label{\detokenize{eip/eam:embedded-atom-model}}\label{\detokenize{eip/eam::doc}}
Notes on the embedded atom model


\subparagraph{Tersoff Potential}
\label{\detokenize{eip/tersoff:tersoff-potential}}\label{\detokenize{eip/tersoff::doc}}
Notes on the tersoff potential.


\subsubsection{Molecular Dynamics}
\label{\detokenize{md/index:molecular-dynamics}}\label{\detokenize{md/index:md}}\label{\detokenize{md/index::doc}}

\paragraph{Overview of Molecular Dynamics}
\label{\detokenize{md/index:overview-of-molecular-dynamics}}

\subparagraph{Introduction to Molecular Dynamics}
\label{\detokenize{md/intro_md:introduction-to-molecular-dynamics}}\label{\detokenize{md/intro_md:intro-md}}\label{\detokenize{md/intro_md::doc}}
Molecular dynamics gives a route to dynamical properties of a system: transport coefficients, time-dependent reponses to perturbations, rheological properties, and
spectra.  Computer simulation serves as a bridge between the microscopic and and macroscopic; theory and experiment.

\begin{figure}[htbp]
\centering
\capstart

\noindent\sphinxincludegraphics[scale=0.5]{{img_md_models}.png}
\caption{Bridge between theory and experiment.}\label{\detokenize{md/intro_md:id1}}\end{figure}


\subparagraph{Atomic interactions}
\label{\detokenize{md/intro_md:atomic-interactions}}
Molecular dynamics simulation consists of a step-by-step numerical solution of the classical equations of motion
\begin{equation*}
\begin{split}m_i \frac{\partial^2 \mathbf{r}}{\partial t^2}  = \mathbf{f}_i\end{split}
\end{equation*}\begin{equation*}
\begin{split}\mathbf{f}_i = - \frac{\partial}{\partial \mathbf{r}_i} \mathcal{U}\end{split}
\end{equation*}
Typically \(\mathcal{U}\) is an empirical atomic potential, although these Energies can be calculated using ab initio techniques such as density functional theory (DFT).
Time integration
—————-


\subparagraph{Periodic boundary conditions}
\label{\detokenize{md/intro_md:periodic-boundary-conditions}}
Small sample sizes means that unless surface effects are of a particular interest, periodic boundary conditions are necessary.
If we consider a system with 1000 atoms arranged in a 10x10x10 cube, nearly half the atoms are on the outer faces, and these surfaces will have a large effect on the measured properties.
Surrounding the cube with replicas of itself takes care of this problem.  Provided the potential range is not too long, we can adopt the minimum image convention that each atom interacts with the nearest atom or image in the periodic array.


\subparagraph{Neighbor lists}
\label{\detokenize{md/intro_md:neighbor-lists}}
This is a more in-depth list of resources both on this website and externally.


\subparagraph{Reading}
\label{\detokenize{md/index:reading}}\begin{itemize}
\item {} 
\phantomsection\label{\detokenize{md/index:id1}}{\hyperref[\detokenize{md/index:allen2004-intro-md}]{\sphinxcrossref{{[}A+04{]}}}} Michael Allen’s \sphinxhref{http://www.physics.iitm.ac.in/~iol/lecture\_notes/allen.pdf}{Introduction to Molecular Dynamics Simulation}

\end{itemize}


\subparagraph{References}
\label{\detokenize{md/index:references}}



\subsubsection{Lattice Dynamics}
\label{\detokenize{ld/index:lattice-dynamics}}\label{\detokenize{ld/index:ld}}\label{\detokenize{ld/index::doc}}
Two frequency-domain methods for predicting phonon frequencies and lifetimes using phonon spectral energy density can be described.

One comes from lattice dynamics.

For a perfect lattice, all vibrational modes are phonon modes, which by definition are delocalized, propagating plane waves.
* Ch 1. \sphinxhref{http://ntpl.me.cmu.edu/pubs/Larkin\_PhDthesis.pdf}{Some dude’s phonon PhD dissertation}
* Julian Gale. \sphinxhref{https://www.epj-conferences.org/articles/epjconf/pdf/2011/04/epjconf\_ft2m2010\_03005.pdf}{Tutorial in Force Field Simulation of Materials with GULP}
* Wolverton Research Group. \sphinxhref{http://wolverton.northwestern.edu/resources/a-practical-guide-to-frozen-phonon-calculations}{A practical guide to frozen phonon calcultions}


\subsubsection{Calculation of Material Properties}
\label{\detokenize{calc_material_properties/index:calculation-of-material-properties}}\label{\detokenize{calc_material_properties/index:calc-material-properties}}\label{\detokenize{calc_material_properties/index::doc}}

\paragraph{Calculating Bulk Properties}
\label{\detokenize{calc_material_properties/calc_bulk_properties:calculating-bulk-properties}}\label{\detokenize{calc_material_properties/calc_bulk_properties::doc}}

\subparagraph{Tools}
\label{\detokenize{calc_material_properties/calc_bulk_properties:tools}}\begin{itemize}
\item {} 
\sphinxhref{http://jp-minerals.org/vesta/en/}{VESTA} is a free 3D visualization program

\end{itemize}

that also allows editing.  It is available for Windows, MacOS, and Linux operating systems.


\paragraph{Calculating Surface Energies}
\label{\detokenize{calc_material_properties/calc_surface_energy:calculating-surface-energies}}\label{\detokenize{calc_material_properties/calc_surface_energy:calc-surface-energy}}\label{\detokenize{calc_material_properties/calc_surface_energy::doc}}

\subparagraph{Notation}
\label{\detokenize{calc_material_properties/calc_surface_energy:notation}}
Vectors and planes in a crystal lattice are described by three-value Miller index notation.  It uses
the indices \(h\), \(l\), and \(k\) as direction parameters.  By definition, the notation \((hlk)\)
denotes a plane which intercepts three points: \(a_1/h\), \(a_2/l\), and \(a_3/k\).

A more convenient definition of the \((hlk)\) family of the planes are the planes orthogonal to the vector \([h,l,k]\).
\begin{itemize}
\item {} 
Coordinates in angle brackets such as \(\langle 100 \rangle\) denote a family of directions that are equivalent due to symmetry operations, such as \([100]\), \([010]\), \([001]\) or the negative of any of those directions.

\item {} 
Coordinates in curly brackets or braces such as \(\langle 100 \rangle\) denote a family of plane normals that are equivalent due to symmetry operations, much the way angle brackets denote a family of directions.

\end{itemize}


\subparagraph{Creating the Structures}
\label{\detokenize{calc_material_properties/calc_surface_energy:creating-the-structures}}
The process for creating slab surfaces has been described in literature.
\begin{itemize}
\item {} 
{[}trans2016\_surface{]} Tran.

\end{itemize}

The strategy to create a surface, it is desirable to have it defined in a orthorhombic simulation (or near orthorhombic simulation cell), with family of planes we are interested in be perpendicular to the z-axis.

To do this we transform the a unit vector in the \([h,l,k]\) direction into a vector in the \([0,0,1]\) direction.
\begin{equation*}
\begin{split}\begin{bmatrix} h \\ l \\ k \end{bmatrix}
\begin{bmatrix} p_{11} & p_{12} & p_{13} \\ p_{21} & p_{22} & p_{23} \\ p_{31} & p_{32} & p_{33} \end{bmatrix}
=
\begin{bmatrix} 0 \\ 0 \\ 1  \end{bmatrix}\end{split}
\end{equation*}
In order to create a surface, it is desireable to have it defined in a orthorhombic simulation cell, with the desired orientation of the surface in the \([0,0,1]\) direction.  I need to find the basis vectors and also the number of basic vector for the interface to be along (110) or (111) plane and the unit cell which I again guess will be cubic. Then I need to repeat these basis vectors along all three axis to generate the crystal with the required orientation.  Given the linear transformation matrix \(\mathbf{M}\) and the translation vector \(\mathbf{t}\).  The new position of a particle will be
\begin{equation*}
\begin{split}x^\prime = \mathbf{M} \mathbf{x} + \mathbf{t}\end{split}
\end{equation*}

\subparagraph{Rotation Matrices}
\label{\detokenize{calc_material_properties/calc_surface_energy:rotation-matrices}}
There are two types of rotational matrices, one which covers polar angle transformation and one that covers Euler angle transformation.

Typically, discussions of transformation matrices are covered under an Euler angle transformation.

Let us define the Euler Angle transformation matrices,

\(\mathbf{R}
The mathematics of rotation matrices are covered in :cite:`evans2001rotations\).

\begin{figure}[htbp]
\centering
\capstart

\noindent\sphinxincludegraphics[width=0.500\linewidth]{{polar_angles}.jpg}
\caption{Polar coordinate representation, image from \label{\detokenize{calc_material_properties/calc_surface_energy:id2}}{\hyperref[\detokenize{calc_material_properties/calc_surface_energy:evans2001rotations}]{\sphinxcrossref{{[}Eva01{]}}}}.}\label{\detokenize{calc_material_properties/calc_surface_energy:id4}}\end{figure}

We know that the \([111]\) vector

Suppose we have the original basis vectors \(\mathbf{a}_1\), \(\mathbf{a}_2\), and \(\mathbf{a}_3\),
The new basis vectors \(\mathbf{a}_1^\prime\), \(\mathbf{a}_2^\prime\), and \(\mathbf{a}_2\prime\).  The the rotation matrix \(\mathbf{P}\) is related to these quantities
\begin{equation*}
\begin{split}\begin{bmatrix} \mathbf{a}_1^\prime & \mathbf{a}_2^\prime & \mathbf{a}_3^\prime \end{bmatrix} =
\begin{bmatrix} \mathbf{a}_1 & \mathbf{a}_2 & \mathbf{a}_3 \end{bmatrix}
\begin{bmatrix} p_{11} & p_{12} & p_{13} \\ p_{21} & p_{22} & p_{23} \\ p_{31} & p_{32} & p_{33} \end{bmatrix}\end{split}
\end{equation*}
\sphinxhref{https://www.researchgate.net/file.PostFileLoader.html?id=560d40475e9d97f2d68b4600\&assetKey=AS\%3A279749973823490\%401443708999090}{Emre Tasci.  How to Prepare an Input File for Surface Calculations.}
\sphinxhref{https://compuphys.wordpress.com/2015/02/10/surface-slabs-mit-vesta/}{Using VESTA to make a surface.}


\subparagraph{Calculation of Suface Energies}
\label{\detokenize{calc_material_properties/calc_surface_energy:calculation-of-suface-energies}}

\subparagraph{Convergence of Surface Energy Calculations}
\label{\detokenize{calc_material_properties/calc_surface_energy:convergence-of-surface-energy-calculations}}\begin{itemize}
\item {} 
The creation and convergence of surface slabs.:cite:\sphinxtitleref{sun2013\_surface\_slabs}

\item {} 
\phantomsection\label{\detokenize{calc_material_properties/calc_surface_energy:id3}}{\hyperref[\detokenize{calc_material_properties/calc_surface_energy:wan1999surface}]{\sphinxcrossref{{[}WFG+99{]}}}}.  Modelling of surface relaxation and stress in fcc metals.

\item {} 
\sphinxtitleref{Surface Calculation example in DFT.\textless{}http://exciting-code.org/lithium-surface-calculations\#toc17\textgreater{}}.  I really like the format of this presentation.  I will use this format in both preparing a DFT exercise, creating a slab convergence workflow, and elucidating the points below.

\end{itemize}


\subparagraph{Converging the width of the slab}
\label{\detokenize{calc_material_properties/calc_surface_energy:converging-the-width-of-the-slab}}

\subparagraph{Converging the amount of the vacuum}
\label{\detokenize{calc_material_properties/calc_surface_energy:converging-the-amount-of-the-vacuum}}

\subparagraph{Converging the kpoint-mesh}
\label{\detokenize{calc_material_properties/calc_surface_energy:converging-the-kpoint-mesh}}
\sphinxhref{http://people.virginia.edu/~lz2n/mse6020/notes/D-partial-fcc.pdf}{Partial dislocations in FCC crystals}


\subparagraph{References}
\label{\detokenize{calc_material_properties/calc_surface_energy:references}}



\paragraph{Calculation of Ideal Crystal Properties}
\label{\detokenize{calc_material_properties/index:calculation-of-ideal-crystal-properties}}\begin{itemize}
\item {} 
Calculation of Energy in DFT

\item {} 
Calculation of Energy in Molecular Dynamics

\end{itemize}


\subparagraph{Structural Minimization}
\label{\detokenize{calc_material_properties/index:structural-minimization}}
Suppose we have an ideal crystal structure with a lattice vectors defined by the matrix \(mathbf{H}\) and a set of vectors with species \(\mathbf{r}_{\alpha,i}\).


\subparagraph{Additional Reading and References}
\label{\detokenize{calc_material_properties/index:additional-reading-and-references}}
Kitchin’s free source on \sphinxhref{http://kitchingroup.cheme.cmu.edu/dft-book/dft.html\#orgheadline10}{Modelling Materials using Density Functional Theory} is an excellent reference on how to automate DFT calcualtions in VASP.


\subsubsection{Potential Development}
\label{\detokenize{pot_dev/index:potential-development}}\label{\detokenize{pot_dev/index::doc}}

\paragraph{Traditional Approach to Potential Parameterization}
\label{\detokenize{pot_dev/param_trad:traditional-approach-to-potential-parameterization}}\label{\detokenize{pot_dev/param_trad::doc}}
Given a configuration of atoms denoted \(\{\mathbf{r}\}\) and an empirical interatomic potential \(V(\mathbf{r}:\mathbf{\theta})\), we can predict a variety of material properties, which we will denote \(\hat{q}_i(\mathbf{\theta})\)


\paragraph{Pareto Estimation of Parameters}
\label{\detokenize{pot_dev/param_pareto:pareto-estimation-of-parameters}}\label{\detokenize{pot_dev/param_pareto::doc}}

\subsubsection{VASP RESOURCES}
\label{\detokenize{vasp/index:vasp-resources}}\label{\detokenize{vasp/index::doc}}
This section contains a variety of different notes on using the plane augmented wave (PAW) density functional theory (DFT) code called VASP.

Compilation of VASP

Running a VASP simulation

VASP cookbooks:
- energy cutoff convergence
- kpoint mesh convergence
\begin{itemize}
\item {} 
kpoint density per reciprocal atom

\end{itemize}
\begin{itemize}
\item {} 
structural minimization

\item {} 
nudged elastic band calculation

\item {} \begin{description}
\item[{calculation of elastic constants}] \leavevmode
in general use, ibrion = 6, isif = 3, potim = 0.015
David Holec, Martin Friák, Jörg Neugebauer, and Paul H. Mayrhofer. Phys. Rev. B 85, 064101
R., Zhu J., Ye H. Calculations of single-crystal elastic constants made simple. Comput. Phys. Commun. 2010;181(3):671\textendash{}675

\end{description}

\end{itemize}

VASP checklist: This is a list of tasks which should be done to determine whether or not your VASP simulation worked.

A VASP simulation requires the following files at a minimum: INCAR, POSCAR, POTCAR, and KPOINTS.


\subsubsection{A non-comprehensive guide to LAMMPS}
\label{\detokenize{lammps/index:a-non-comprehensive-guide-to-lammps}}\label{\detokenize{lammps/index::doc}}
One of the problems in molecular dynamics simultions is that a robust set of simulations in which to learn from aren’t generally available all in one location.  These scripts were collected from different parts of the internet.  I have tried to provide credit to where I find these simulations.


\subsubsection{GULP}
\label{\detokenize{gulp/index:gulp}}\label{\detokenize{gulp/index:gulp-index}}\label{\detokenize{gulp/index::doc}}
The General Utility Lattice Program (GULP) is a progam which excels in the implementation of lattice dynamics solutions.

GULP is extremely well documented with the \sphinxtitleref{GULP manuals \textless{}https://nanochemistry.curtin.edu.au/gulp/help/manuals.cfm\textgreater{}}
\sphinxhref{https://nanochemistry.curtin.edu.au/gulp/help/examples.cfm}{GULP example}

The purpose of these notes is to provide an introduction to the physical properties of solids, which are of extraordinarily important in the modern world.  It focuses upon fundamental, unifying concepts in solid-state physics but with a computational approach to under the properties of nuclei and electrons in solids rather than the typical experimental approach found in most solid state physics books.  These notes look to provide a practical introduction to computational materials to establish the basic principles, to describe phenomena responsible for the importance of solids in science and technology, while at the same time orienting the discussion toward understanding computational materials techniques and using them for explanatory purposes as well as to predict behavior.

The purpose of these notes is not intended to be exhaustive there exists many existing textbooks on computational materials as well the theory in materials science.  As a result, I have attempted to provide a fairly extensive bibliography which include recommended textbooks, as well as citation to the original journal articles.  Since this is an online reference, I intend to develop this website as a series of monographs and tutorials.


\subsection{Introduction to Computational Tools}
\label{\detokenize{computational_simulation_tools:introduction-to-computational-tools}}

\subsubsection{General Reference Books for Computational Material Science}
\label{\detokenize{computational_simulation_tools:general-reference-books-for-computational-material-science}}\begin{itemize}
\item {} 
\phantomsection\label{\detokenize{computational_simulation_tools:id1}}{\hyperref[\detokenize{computational_simulation_tools:lee2016-book-comp-mse}]{\sphinxcrossref{{[}Lee16{]}}}} Lee, J.G.  Computational Materials Science: an Introduction. 2016. CRC Press.

\end{itemize}


\subsubsection{Crystallography}
\label{\detokenize{computational_simulation_tools:crystallography}}
{\hyperref[\detokenize{crystallography/index:crystal-lattice}]{\sphinxcrossref{\DUrole{std,std-ref}{The Crystal Lattice}}}}

{\hyperref[\detokenize{crystallography/reciprocal_lattice:reciprocal-lattice}]{\sphinxcrossref{\DUrole{std,std-ref}{Reciprocal Lattice}}}}

For building crystal structres in a programatic way, the atomistic simulation environment is well-developed package.

Atomic Simulation Environment.  \sphinxhref{https://wiki.fysik.dtu.dk/ase/ase/build/build.html\#common-bulk-crystals}{Building Things}


\subsubsection{Density Functional Theory}
\label{\detokenize{computational_simulation_tools:density-functional-theory}}
{\hyperref[\detokenize{dft/dft_intro:dft-intro}]{\sphinxcrossref{\DUrole{std,std-ref}{Introduction to Density Functional Theory, by Eugene Ragasa}}}}

\sphinxhref{https://www.vasp.at}{VASP}

\sphinxtitleref{pypospack VASP examples \textless{}pypospack/examples/vasp/index\textgreater{}}


\subsubsection{Molecular Dynamics}
\label{\detokenize{computational_simulation_tools:molecular-dynamics}}
{\hyperref[\detokenize{md/intro_md:intro-md}]{\sphinxcrossref{\DUrole{std,std-ref}{Introduction to Molecular Dynamics, by Eugene Ragasa}}}}

\sphinxhref{https://udel.edu/~arthij/MD.pdf}{Introduction to Molecular Dynamics Simulation, by Micheal P. Allen}

\sphinxhref{http://lammps.sandia.gov}{LAMMPS}


\subsubsection{Lattice Dynamics}
\label{\detokenize{computational_simulation_tools:lattice-dynamics}}
The study of vibrations of atoms inside crystals, lattice dynamics, is basic to many fields of study in solid state physics and materials science.

\sphinxhref{https://www.amazon.com/Introduction-Lattice-Dynamics-Cambridge-Chemistry/dp/0521398940}{Introduction to Lattice Dynamics, by Martin T. Dove} is a well written book accessible to advanced undergraduates, graduate students, and reserach workers looking to understand phonons from an approachable but fairly rigorous perspective.

\sphinxhref{https://gulp.curtin.edu.au/gulp/}{gulp} is software written in Fortran which can do a variety of molecular dynamics type tasks, but really excells in calculated phonons and phonon density of states for a variety of empirical interatomic potentials.

\sphinxhref{https://atztogo.github.io/phonopy/}{phonopy}


\subsection{Material Properties and Defects in Crystals}
\label{\detokenize{computational_simulation_tools:material-properties-and-defects-in-crystals}}
These links are useful for building defect stuctures.
\begin{itemize}
\item {} 
Atomic Simulation Environment.  \sphinxhref{https://wiki.fysik.dtu.dk/ase/ase/lattice.html}{General crystal structure}

\end{itemize}


\subsection{Development of Interatomic Potentials}
\label{\detokenize{computational_simulation_tools:development-of-interatomic-potentials}}

\subsection{References}
\label{\detokenize{computational_simulation_tools:references}}



\section{pypospack}
\label{\detokenize{pypospack/index:pypospack}}\label{\detokenize{pypospack/index::doc}}

\section{pyflamestk}
\label{\detokenize{pyflamestk/index:pyflamestk}}\label{\detokenize{pyflamestk/index::doc}}
pyflamestk is the name of a software package developed at the University of Florida as a graduate student.  I intend to incrementally build upon the software platform to make it a robust tool for developing empirical interatomic potentials useful in molecular dynamics.  At some point, I hope to publish this software.


\subsection{pyflamestk User Guide}
\label{\detokenize{pyflamestk/usr/index:pyflamestk-user-guide}}\label{\detokenize{pyflamestk/usr/index::doc}}

\subsection{pyflamestk Developers Guide}
\label{\detokenize{pyflamestk/dev/index:pyflamestk-developers-guide}}\label{\detokenize{pyflamestk/dev/index::doc}}

\subsection{pyflamestk API documentation}
\label{\detokenize{pyflamestk/api/index:pyflamestk-api-documentation}}\label{\detokenize{pyflamestk/api/index::doc}}

\section{Tutorials and Reference}
\label{\detokenize{tutorials:tutorials-and-reference}}\label{\detokenize{tutorials::doc}}
These tutorials are more a combination of scripts and notes which I have written over time.  The notes may not be particularly coherent.  I will slowly build up some of these notes into some type of coherent structure as time permits.


\subsection{Sphinx}
\label{\detokenize{tutorials:sphinx}}
This website is build entirely using.  Sphinx is a document generator which converts reStructuredText files into HTML websites and other formats, e.g. LaTeX (PDF) and man pages.  Since this website is a really a combination of different Sphinx documents, I am slowly learning things in Sphinx and some tricks and tips for how I use Sphinx.


\subsubsection{Sphinx Tutorial}
\label{\detokenize{sphinx_tutorial/index:sphinx-tutorial}}\label{\detokenize{sphinx_tutorial/index::doc}}

\paragraph{Getting Sphinx in Anaconda}
\label{\detokenize{sphinx_tutorial/index:getting-sphinx-in-anaconda}}
I use the Anaconda distribution by Continuum Analytics


\paragraph{Building a Personal Website using Sphinx}
\label{\detokenize{sphinx_tutorial/index:building-a-personal-website-using-sphinx}}
\sphinxhref{https://blog.shichao.io/2013/03/19/create\_a\_personal\_website\_with\_sphinx.html}{Creating a Personal Website with sphinx}


\paragraph{Autodocumenting python using Sphinx}
\label{\detokenize{sphinx_tutorial/index:autodocumenting-python-using-sphinx}}

\section{Mathematics}
\label{\detokenize{math/index:mathematics}}\label{\detokenize{math/index::doc}}

\subsection{An Computational Apprach to Differnential Equations}
\label{\detokenize{math/differential_equations:an-computational-apprach-to-differnential-equations}}\label{\detokenize{math/differential_equations::doc}}

\subsubsection{Existance and Uniqueness of Solutions}
\label{\detokenize{math/differential_equations:existance-and-uniqueness-of-solutions}}
The existence and uniqueness theorems in differential equations helps us determine if a solution to differential equation and if that solution is unique.

To motivate this example let us consider existance and uniqueness of solutions for simple algebraic systems.  Consider \(3x+4=0\) has a solution for \(x\) exists, \(x=4/3\), which is also a unique equation.  However, the quadtratic equation \(x^2=4\) has the solution \(x=\{-2,2\}\).  In this situation, the quadratic equatihas solutions that exist, but those solutions are not unique.

Existance and Uniqueness Theorem.  Given the initial boundary problem (IVP)
\begin{equation*}
\begin{split}\frac{dy}{dt} = f(t,y), y(t_0)=y_0\end{split}
\end{equation*}\begin{enumerate}
\item {} 
Exitance.  If \(f(t,y)\) is continuous in the region surrounding the initial value, \(y(t_0)=y-0\), then we can define a rectangular region \(R\),

\end{enumerate}
\begin{equation*}
\begin{split}R = \bigl \{(t,y)|a<t<b,c<y<d \bigr \}\end{split}
\end{equation*}
that contains the point \((t_0,y_0)\), then the IVP has a unique solution, y(t)
\begin{enumerate}
\setcounter{enumi}{1}
\item {} 
Uniqueness.  If \(\partial f / \partial y\) is continuous in the region \(R\) then the IVP has a unique solution, y(t)

\end{enumerate}


\subsection{Partial Differential Equations}
\label{\detokenize{math/periodic_boundary_conditions:partial-differential-equations}}\label{\detokenize{math/periodic_boundary_conditions::doc}}

\subsubsection{Periodic Boundary Conditions}
\label{\detokenize{math/periodic_boundary_conditions:periodic-boundary-conditions}}\begin{itemize}
\item {} 
\sphinxhref{http://www.eng.famu.fsu.edu/~dommelen/pdes/style\_a/svcex.html}{An Example with Periodic Boundary Conditions}

\end{itemize}


\subsection{Fourier Transform}
\label{\detokenize{math/fourier_transform:fourier-transform}}\label{\detokenize{math/fourier_transform::doc}}
James O’Brien “\sphinxhref{http://www.phys.uconn.edu/~obrien/index\_files/fourier.pdf}{Introduction to Fourier Transforms for Physicists.}”

Let us denote, the Fourier transform of the function \(f\) is denoted by the hat symbol: \(\hat{f}\).
\begin{equation*}
\begin{split}\hat{f}(\xi) = \int_{-\infty}^{\infty} f(x) exp(-2 \pi i x \xi) dx\end{split}
\end{equation*}
for any real number \(\xi\)

Under suitable conditions, \(f\) can be determined from \(\hat{f}\) through the inverse transform
\begin{equation*}
\begin{split}f(x) = \int_{\infty}^{\infty} \hat{f}(\xi) exp(-2 \pi i x \xi) d \xi\end{split}
\end{equation*}

\subsection{Heat Equation}
\label{\detokenize{math/heat_equation:heat-equation}}\label{\detokenize{math/heat_equation::doc}}

\subsection{Mathematics of the Wave Equation}
\label{\detokenize{math/wave_equation:mathematics-of-the-wave-equation}}\label{\detokenize{math/wave_equation::doc}}

\subsubsection{Wave Equation in 1D}
\label{\detokenize{math/wave_equation:wave-equation-in-1d}}
Let \(u=u(x,t)\) denote the location of a particle subject to an oscillation.


\subsection{Green’s Function}
\label{\detokenize{math/greens_function:green-s-function}}\label{\detokenize{math/greens_function::doc}}
\sphinxhref{http://www.dtic.mil/dtic/tr/fulltext/u2/717870.pdf}{An Introduction to Greens Functions} by Thomas Eisler is a free monograph through the United States Department of Defense.


\section{Message Passing Interface}
\label{\detokenize{mpi/index:message-passing-interface}}\label{\detokenize{mpi/index::doc}}
Messaging Passing Interface (MPI)

An MPI program is launched as separate processes, each with their own address space, and requires partitioning data across tasks.


\subsection{MPI4py}
\label{\detokenize{mpi/index:mpi4py}}
MPI4Py provides an interface similar to the MPI-2 standard C++ interface.

tutorial\_mpi4py\_1.rst

\sphinxurl{https://portal.tacc.utexas.edu/c/document\_library/get\_file?uuid=be16db01-57d9-4422-b5d5-17625445f351\&groupId=13601}


\section{Quantum Mechanics}
\label{\detokenize{qm/index:quantum-mechanics}}\label{\detokenize{qm/index::doc}}
Joel Franklin.  Lecture Notes for Quantum Mechanics I.  \sphinxurl{http://academic.reed.edu/physics/courses/P342.S10/Physics342}
\sphinxurl{http://www.reed.edu/physics/courses/P342.S10/Physics342/page1/files/Lecture.2.pdf}


\chapter{Indices and tables}
\label{\detokenize{index:indices-and-tables}}\begin{itemize}
\item {} 
\DUrole{xref,std,std-ref}{genindex}

\item {} 
\DUrole{xref,std,std-ref}{modindex}

\item {} 
\DUrole{xref,std,std-ref}{search}

\end{itemize}

\begin{sphinxthebibliography}{WFG+99}
\bibitem[Gia02]{\detokenize{Gia02}}{\phantomsection\label{\detokenize{crystallography/perfect_crystal:giacovazzo2002-book-crystal}} 
Carmelo Giacovazzo. \sphinxstyleemphasis{Fundamentals of crystallography}. Volume 7. Oxford university press, USA, 2002.
}
\bibitem[HH01]{\detokenize{HH01}}{\phantomsection\label{\detokenize{crystallography/perfect_crystal:hammond2001-book-crystal}} 
Christopher Hammond and Christopher Hammond. \sphinxstyleemphasis{Basics of crystallography and diffraction}. Volume 214. Oxford, 2001.
}
\bibitem[San69]{\detokenize{San69}}{\phantomsection\label{\detokenize{crystallography/perfect_crystal:sands1969-book-crystals}} 
Donald E Sands. \sphinxstyleemphasis{Introduction to crystallography}. Courier Corporation, 1969.
}
\bibitem[San02]{\detokenize{San02}}{\phantomsection\label{\detokenize{crystallography/perfect_crystal:sands2002-book-crystals-2}} 
Donald E Sands. \sphinxstyleemphasis{Vectors and tensors in crystallography}. Courier Corporation, 2002.
}
\bibitem[San69]{\detokenize{San69}}{\phantomsection\label{\detokenize{crystallography/index:sands1969-book}} 
Donald E Sands. \sphinxstyleemphasis{Introduction to crystallography}. Courier Corporation, 1969.
}
\bibitem[San69]{\detokenize{San69}}{\phantomsection\label{\detokenize{crystallography/index:id3}} 
Donald E Sands. \sphinxstyleemphasis{Introduction to crystallography}. Courier Corporation, 1969.
}
\bibitem[Cap06]{\detokenize{Cap06}}{\phantomsection\label{\detokenize{dft/dft_intro:capelle2006-dft}} 
Klaus Capelle. A bird’s-eye view of density-functional theory. \sphinxstyleemphasis{Brazilian Journal of Physics}, 36(4A):1318\textendash{}1343, 2006.
}
\bibitem[Gri16]{\detokenize{Gri16}}{\phantomsection\label{\detokenize{dft/dft_intro:griffiths1995-book-qm}} 
David J Griffiths. \sphinxstyleemphasis{Introduction to quantum mechanics}. Cambridge University Press, 2016.
}
\bibitem[SN14]{\detokenize{SN14}}{\phantomsection\label{\detokenize{dft/dft_intro:sakurai2014-book-qm}} 
Jun John Sakurai and Jim J Napolitano. \sphinxstyleemphasis{Modern quantum mechanics}. Pearson Higher Ed, 2014.
}
\bibitem[Sha12]{\detokenize{Sha12}}{\phantomsection\label{\detokenize{dft/dft_intro:shankar2012-book-qm}} 
Ramamurti Shankar. \sphinxstyleemphasis{Principles of quantum mechanics}. Springer Science \& Business Media, 2012.
}
\bibitem[A+04]{\detokenize{A+04}}{\phantomsection\label{\detokenize{md/index:allen2004-intro-md}} 
Michael P Allen and others. Introduction to molecular dynamics simulation. \sphinxstyleemphasis{Computational soft matter: from synthetic polymers to proteins}, 23:1\textendash{}28, 2004.
}
\bibitem[Eva01]{\detokenize{Eva01}}{\phantomsection\label{\detokenize{calc_material_properties/calc_surface_energy:evans2001rotations}} 
Philip R Evans. Rotations and rotation matrices. \sphinxstyleemphasis{Acta Crystallographica Section D: Biological Crystallography}, 57(10):1355\textendash{}1359, 2001.
}
\bibitem[WFG+99]{\detokenize{WFG+99}}{\phantomsection\label{\detokenize{calc_material_properties/calc_surface_energy:wan1999surface}} 
Jun Wan, YL Fan, DW Gong, SG Shen, and XQ Fan. Surface relaxation and stress of fcc metals: cu, ag, au, ni, pd, pt, al and pb. \sphinxstyleemphasis{Modelling and Simulation in Materials Science and Engineering}, 7(2):189, 1999.
}
\bibitem[Lee16]{\detokenize{Lee16}}{\phantomsection\label{\detokenize{computational_simulation_tools:lee2016-book-comp-mse}} 
June Gunn Lee. \sphinxstyleemphasis{Computational materials science: an introduction}. CRC press, 2016.
}
\end{sphinxthebibliography}



\renewcommand{\indexname}{Index}
\printindex
\end{document}